\input{suhw.tex}
\usepackage{graphicx,amssymb,amsmath,enumerate}
\usepackage{courier}
\usepackage{color}
\usepackage{listings}
\usepackage{fancyvrb}
\usepackage{stmaryrd}

\definecolor{dkgreen}{rgb}{0,0.6,0}
\definecolor{gray}{rgb}{0.5,0.5,0.5}
\lstset{language=Python,
	frame=lines,
   basicstyle=\ttfamily\fontsize{8}{12}\selectfont,
   keywordstyle=\color{blue},
   commentstyle=\color{red},
   stringstyle=\color{dkgreen},
   numbers=left,
   numberstyle=\tiny\color{gray},
   stepnumber=1,
   numbersep=10pt,
   backgroundcolor=\color{white},
   tabsize=2,
   showspaces=false,
   showstringspaces=false,
   lineskip=-3.5pt }
\oddsidemargin 0in
\evensidemargin 0in
\textwidth 6.5in
\topmargin -0.5in
\textheight 9.0in

\begin{document}

\normaldoc{CS276: Information Retrieval and Web Search}{Spring 2013}{Programming Assignment 1}{Botao Hu (botaohu), Jiayuan Ma (jiayuanm)}{\today}

\pagestyle{myheadings}  % Leave this command alone

\section{Design and Implementation}

\subsection{Task 1}
\subsection{Task 2}
\subsection{Extra credit}


\section{Answers}

\begin{enumerate}[(a)]
\item 
In this PA we asked you to use each sub-directory as a block and build index
for one block at a time. Can you discuss the tradeoff of different sizes of
blocks? Is there a general strategy when we are working with limited memory
but want to minimize indexing time?

\item Is there a part of your indexing program that limits its scalability to larger
datasets? Describe all the other parts of the indexing process that you can
optimize for indexing time/scalability and retrieval time.

\item 
Any more ideas of how to improve indexing or retrieval performance?
\end{enumerate}



\end{document}

